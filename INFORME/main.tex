\documentclass[10pt,a4paper]{article}
%\usepackage[english,spanish]{babel}
\usepackage{indentfirst}
\usepackage{anysize} % Soporte para el comando \marginsize
%\marginsize{1.5cm}{1.5cm}{0.5cm}{1cm}
\marginsize{2,5cm}{1,8cm}{4cm}{1,7cm}
\usepackage[psamsfonts]{amssymb}
\usepackage{amssymb}
\usepackage{amsfonts}
\usepackage{amsmath}
\usepackage{multirow} % para las tablas
\usepackage{amsthm}
\usepackage{stackrel}
\usepackage{graphicx}
\usepackage[colorinlistoftodos]{todonotes}
%Color a las referencias
\usepackage[colorlinks=true, allcolors=blue]{hyperref}
\usepackage[spanish]{babel}
\selectlanguage{spanish}
\usepackage[utf8]{inputenc} 
\usepackage{multicol}
\renewcommand{\thepage}{}
\columnsep=7mm

%%%%%%%%%%%%%%%%%%%%%%%%%%%%%%%%%%%%%%%%
\newtheorem{definicion}{Definici\'on}[section]
\newtheorem{teorema}{Teorema}[section]
\newtheorem{prueba}{Prueba}[section]
\newtheorem{prueba*}{Prueba}[section]
\newtheorem{corolario}{Corolario}[section]
\newtheorem{observacion}{Observaci\'on}[section]
\newtheorem{lema}{Lema}[section]
\newtheorem{ejemplo}{Ejemplo}[section]
\newtheorem{solucion*}{Soluci\'on}[section]
\newtheorem{algoritmo}{Algoritmo}[section]
\newtheorem{proposicion}{Proposici\'on}[section]

\linespread{1.4} \sloppy

\newcommand{\R}{\mathbf{R}}
\newcommand{\N}{\mathbf{N}}
\newcommand{\C}{\mathbb{C}}
\newcommand{\Lr}{\mathcal{L}}
\newcommand{\fc}{\displaystyle\frac}
\newcommand{\ds}{\displaystyle}

\DeclareMathOperator{\Dom}{Dom}

%%%%%%%%%%%%%%%%%%%%%%%%%%%%%%%%%%%%%%%%

\renewcommand{\thefootnote}{\fnsymbol{footnote}}
\usepackage{url}
\usepackage{hyperref}

\begin{document}
\begin{center}
 {\Large \textbf{PREDICCIÓN DE LA CONCENTRACIÓN DEL PRODUCTO EN UNA REACCIÓN QUÍMICA}}
\end{center}
\begin{center}
 Lázaro-Camasca$^{1}$, Ponce-Pinedo$^{2}$, Sarria-Palacios$^{3}$, Loayza-Pizarro$^{4}$\vskip5pt
 {\it Facultad de Ciencias$^1$, Universidad Nacional de Ingenier\'{\i}a$^1$\\}\vskip5pt
 Email: elazaroc@uni.pe$^{1}$, victor.ponce.p@uni.pe$^{2}$,esarriap@uni.pe$^{3}$, fernando.loayza.p@uni.pe$^{4}$
\end{center}
%\maketitle 
\vspace*{0.4cm}
\begin{abstract}


En muchos problemas cientificos se necesita obtener datos a partir de casos experimentales.


\noindent Conocer estas concentraciones ayuda a muchos profesionales. Los químicos y biólogos miden las cantidades agentes contaminantes para determinar los niveles de \textbf{contaminación en el ambiente}. En la industria farmacéutica los laboritaristas miden las cantidades de sustancias necesarias para preparar medicamentos; todas estas de concentración determinada y de cuya exacta preparación \textbf{depende de la vida y la pronta recuperación de cientos de miles de enfermos}. En las industrias de bebidas gaseosas los ingenieros miden las cantidades de edulcorantes, cafeína, ácido fosfórico, entre otros, con el propósito de que estas sean gratas al paladar, refrescantes y comercialmente rentables, \textbf{ingrementando sus ingresos economicos}.

 

\end{abstract}

\begin{quotation}
	{\small
		\noindent\textbf{Palabras Clave:} \\ 
	Interpolación, Predicción química, Concentración, Contaminación ambiental.
	}
\end{quotation}

\renewcommand{\abstractname}{Abstract}
\begin{abstract}
	\noindent Mathematics is the pillar of many disciplines such as physics, chemistry, computing, biology, engineering. In particular, the \textbf{Polynomial Interpolation} method allows us to obtain data from others, in this case the data becomes the chemical concentrations. Knowing the knowledge of these concentrations helps many professionals. Chemists and biologists measure the amounts of carbon monoxide and dioxide, sulfur dioxide and other pollutants to determine the levels of pollution in the environment. The laboratoristas that work in the pharmaceutical industry measure the quantities of substances necessary to prepare nasal, ophthalmic, sedative, analgesic, antispasmodic, moisturizing solutions; all of these of determined concentration and whose exact preparation \textbf{depends on the life and the quick recovery of hundreds of thousands of patients}. In the soft drink industries engineers measure the amounts of sweeteners, caffeine, phosphoric acid, among others, with the purpose that these are pleasant to the palate, refreshing and commercially profitable,\textbf{increase the income significantly to the industries}.
	
\end{abstract}


\begin{quotation}
{\small
	\noindent \textbf{Keywords:} \\ 
		Interpolation, chemical prediction, concentration, environmental pollution, industrial income.\\
	}
\end{quotation}


\pagebreak

\begin{multicols}{2}
\begin{center}
{\large \bf 1. INTRODUCCI\'ON}
\end{center}

En las reacciones quimicas la velocidad de la reacción dependen de diversos parametros tales como la temperatura, la presión, la concentracion de los reactivos estos al ser va


Conocer la concentración de las soluciones es muy importante porque gracias a ello se puede establecer las cantidades de soluto y solvente presentes en una solución, muchos profesionales tienen que medir, necesariamente, una de las siguientes magnitudes físicas: Masa (m), volumen (v) y cantidad de sustancia (n). \\
 
Por ello presente trabajo describe la obtención de una función que describirá la concentración de producto de una reacción química en función del tiempo. Esta función será calculada mediante métodos de interpolación polinomial utilizando puntos que han sido obtenidos de manera experimental.

\begin{center}
{\large \bf 2. CONCEPTOS PREVIOS}
\end{center}
%==============================================================================
\noindent \textbf{PARTE QUÍMICA}:

\begin{itemize}

	\item \textbf{Reacciones Químicas:} Es un proceso en el que dos o más sustancias (reactantes) se transforman en una o más sustancias (productos). Estas reacciones deben satisfacer la ley de la conservación de la materia.
    \begin{center}
		$aA  +  bB  ->  cC  +  dD$
    \end{center}
	\item \textbf{Velocidad Instantánea de Reacción}: Expresa el cambio de la concentración de un reactivo o de un producto por unidad de tiempo (concentración/tiempo).
	{\scriptsize
	         v = -1/a * d[A]/d[t] =-1/b * d[B]/d[t] =1/c * d[C]/d[t] =1/d* d[D]/d[t] 
	}

	\item \textbf{ORDEN DE LAS REACCIONES}
	\item \textbf{Reacciones de Orden Cero}: Son aquellas en que la velocidad no depende de la concentración de los reactivos. Entonces para: $A -> B$, la ley de velocidad será: $v=k[A]^0=k$. Donde $k$ es la constante de velocidad y su unidad es \[Mt^{-1}\] Por tanto:
	$v=-d[A]/dt=k$.
	Entonces:  \[[A]=-kt+[A]_0\]

	\item \textbf{Reacciones de Primer Orden}: Para la reacción genérica de primer orden: $A \Rightarrow B$, la ley de velocidad será: $v=k[A]^1$, la unidad de $k$ es $t^{-1}$. Entonces se tiene:
	$v=-d[A]/dt=k*[A]$\\
	Resolviendo:   \[ln[A]=-kt+ ln[A]_0\]
	
	\item \textbf{Reacciones de Segundo Orden}: Para la reacción genérica de segundo orden: $A ->B$, la ley de velocidad será: $v=k[A]^2$, la unidad de $k$ es \[M^{-1} t^{-1}\]
	Por tanto: 
	$v=-d[A]/dt=k*[A]^2$
	Entonces: \[1/[A] =-kt+  1/[A]_0 \]


\end{itemize}
%==============================================================================
\noindent \textbf{PARTE NUMÉRICA}:	
\begin{itemize}
	\item \textbf{INTERPOLACIÓN POLINÓMICA}
	
	Dados n+1 puntos $(x_0,y_0), (x_1,y_1), … , (x_n,y_n)$ con $x_i$\not=$x_j$, si $i$\not=$j$ $ existe un único polinomio $ $P_n (x)$ de $grado $\leq$ n$ $ tal que $ $P(x_i )=y_i  \forall i=0,1,…,n$.
	\item \textbf{Datos de Interpolación}: 
	${(x_k,f_k )} $$, k=0,1,…..,n$.
	Conocemos los valores de una función $f_k=f(x_k)$, en $n+1$ puntos distintos $x_k$, de un intervalo $[a,b]$.
	
	\item \textbf{Funciones Interpolantes}: Polinomios de grados menor o igual que n.
	$P(x)=a_0+a_1 x+⋯⋅+a_n x^n$

	\item \textbf{Problema de Interpolación}: Determinar los coeficientes $a0, a1,…, an$ para que cumplan las condiciones de interpolación: \[P(x_k )= f_k,k=0,1,….,n\]\\

	\item \textbf{MÉTODOS DE INTERPOLACIÓN POLINÓMICA}

	\item \textbf{Método de Lagrange}:
	Consiste en calcular previamente los polinomios  $L_i (x_i ),i=0,1,….,n$, llamados polinomios de Lagrange, que verifican:

	$L_i (x_i )=1,L_i (x_j )=0,i\not=j$.
	Estos polinomios vienen dados por la expresión:

	$L_i (x)=\prod_{j=0,j\not=i}^{n}(x-x_j)/(x_i-x_j )$.
	El polinomio de interpolación se escribe de la forma:
	
	$ P(x)=f_0 L_0 (x)+f_1 L_1 (x)+⋯⋅+f_n L_n (x)=\sum_{i=0}^{n}f_i L_i (x) $
	
	\item \textbf{Método Serie de Potencias}:
	Este método construye un polinomio interpolante de orden n $P(x)=a_0+a_1*x+...+a_n*x^n $ para n+1 puntos dados, reemplazando los puntos dados en el polinomio y resolviendo el sistema de ecuaciones generado, hallando de esta manera sus coeficientes.
	
	\item \textbf{Método de Newton (Diferencias Divididas)}:
	Consiste en escribir el polinomio de interpolación en la forma:
	
	$P(x)=A_0+A_1 (x-x_0 )+A_2 (x-x_0 (x-x_1 ))…+A_(n-1) (x-x_0 (x-x_1 )…(x-x_(n-1) )) $\\
	Los coeficientes  $A_k$, que se denominan diferencias divididas de la función f en los puntos $x_0,x_1,…, x_k$, se denotan por $A_k=f [x_0,x_1,…..,x_0]$ y se genera de forma recursiva mediante la fórmula: \\
	$f [x_0,x_1,…..,x_0 ]=(f [x_1,x_2,…..,x_k ]-f [x_0,x_1,…..,x_{k-1}]  )/(x_k-x_0 )$
	Partiendo de $f [x_0]=f (x_0)$.\\
	Con esta notación, el polinomio de interpolación puede escribirse como:\\
	$P(x)= f [x_0 ]+f [x_0,x_1]( x-x_0)+ f [x_0,x_1,x_2]( x-x_0) ( x-x_2)+….+ f [x_0,x_1,…,x_n]( x-x_0) ( x-x_1)….. ( x-x_{n-1})$.
	\item \textbf{INTERPOLACIÓN POLINOMIAL POR PARTES}\\
	El uso de polinomios de interpolación de alto grado puede producir errores grandes debido al alto grado de oscilación de este tipo de polinomios. Para evitar este problema se busca aproximar    la  función  desconocida  en  intervalos  pequeños  usando  polinomios  de  grado bajo. El caso más común de la interpolación por partes es usar polinomios cúbicos. 
	\item \textbf{Interpolacion con Funciones Splines}:
	Sea $\triangle$ una partición del intervalo $[a.b]$.
	$\triangle _:  a=x_0<x_1<...<x_n=b$.
	Un spline es una función polinómica a trozos en cada uno de los intervalos $[x_i,x_{i+1}]$ de la partición.
	Notaremos por $S_m^k (\triangle)$ al conjunto de funciones de clase $k$ que son polinomios a trozos de grado $m$ en cada uno de los intervalos de partición:
	
	$S_n^k (\triangle)=\{ s\in C_m^k  [a,b] , s \|[x_i,x_{i+1} ]\in P_m\}$,
	
	donde $P_m$ denota el conjunto de polinomios de grado a lo sumo m.
	
	Notaremos:
	$\triangle={x_0,x_1,….,x_n  }$ la partición dado por los nodos.
	$h_i=x_{i+1}-x_i$ , a la amplitud del intervalo $[x_i,x_{i+1}]$.
	$f_i$ a los valores de la función que queremos interpolar.
	$m_i=(f_{i+1}-f_i)/h_i$ , a la pendiente de la curva en el intervalo $[x_i,x_{i+1}]$.
	
	\textbf{Interpolacion con splines cúbicos}: $S_3^2 (\triangle)$
	
	Funciones interpolantes: Splines en $S_3^2 (\triangle)$ se trata de funciones de clase 2 definidas a trozos mediante polinomios de grado 3 en cada intervalo de la partición. 
	Problema de interpolación: Determinar un spline $s\in S_3^2 (\triangle)$ tal que:
	$s (x_i )=f_i, i=0,1,….n$, y necesitamos imponer además dos condiciones adicionales en los puntos de frontera $x_0$ y $x_n$. Por ejemplo si el spline satisface  $s^" (x_0 )=0$ y $s^" (x_n )=0$ se dice que es un spline cubico natural.
	
	\item \textbf{REGRESIÓN POR MÍNIMOS CUADRADOS}\\
	En este método se pretende trazar la recta que más se acerque al conjunto de datos dado, expresada matemáticamente como: \[Y=C_1X+C_2+error\]
	Los valores $C_1, C_2$ y el error, se pueden calcular de las siguientes maneras:
	{\scriptsize \[C_1=(n\sum X_i*Y_i -\sum X_i\sum Y_i)/(n\sum X_i^{2}-(\sum X_i)^{2})\]}
	\[C_2=\sum Y_i/n - C_1\sum X_i/n\]
	\[error=(\sum Y_i-C_2-C_1X_i)^{2}\]
	El error se da en términos de la suma de los cuadrados de la diferencia entre el valor muestral y el valor calculado con la recta de regresión.\\
	Si los datos no son muy exactos o tienen asociado un error entonces la mejor manera es establecer una sola curva que represente la tendencia general de los datos observados.
	
	\item \textbf{ESTIMACIÓN DEL ERROR}\\
	Sea $f\in C^{n+1}[a,b]$. Sea $P_n(x)$ el polinomio de grado $\leq n$ que interpola f en los n+1 puntos (distintos) $x_0, x_1, ..., x_n$ en el intervalo [a,b]. Para cada valor fijo $x\in [a,b]$ existe $\xi (x) \in <a,b>$ tal que:
	
	 $f(x)-P_n(x) = f^{n+1}(\xi (x))/(n+1)! * (x-x_0)(x-x_1)...(x-x_n)$

\end{itemize}

\begin{center}
	{\large \bf 3. ANÁLISIS}
	
\end{center}

\noindent En una reacción quiímica, la concentración del producto $C_B$ cambia con el tiempo como se indica en la siguiente tabla:


\begin{center}
	\begin{tabular}{ |c|c| }
		\hline
		$t$ & $C_B$ \\ \hline
		0.00 & 0.00 \\ \hline
		0.10 & 0.30 \\ \hline
		0.40 & 0.55 \\ \hline
		0.60 & 0.80 \\ \hline
		0.80 & 1.10 \\ \hline
		1.00 & 1.15 \\ \hline
	\end{tabular}
\end{center}

\textbf{Calcule la concentración $C_B$ cuando $t = 0.82$}\\

		
\noindent Para resolver el problema se implementará una \textbf{interfaz gráfica} donde el usuario pueda elegir el método numérico para resolver el problema, además podrá hallar la $C_B$ en el cualquier tiempo $t$.  

\noindent Con esto es usuario podrá comparar la eficiencia de los diferentes métodos numéricos ver cual es el mejor y el peor, también podrá insertar sus datos para un mayor aprendizaje, ya que la \textbf{eficiencia de los métodos} depende del problema y por ende de los datos.

\begin{center}
{\large \bf 4. OBSERVACIONES}
\end{center}

\begin{center}
{\large \bf 5. CONCLUSIONES}
\end{center}

\begin{center}
{\large \bf Agradecimientos}
\end{center}
Los autores agradecen a las autoridades de la Facultad de Ciencias de la Universidad Nacional de 
Ingenier\'{\i}a por su apoyo.
%\begin{center}
%{\large \bf Apendice: }
%\end{center}

\end{multicols}
\newpage

\begin{center}
 -----------------------------------------------------------------------------
\end{center}
\begin{multicols}{2}
\begin{list}{}{\setlength{\topsep}{0mm}\setlength{\itemsep}{0mm}%
\setlength{\parsep}{0mm}\setlength{\leftmargin}{4mm}}
%
%------------------------------------- References --------------------
\small
\item[1.] Jaan Kiusalaas, \textit{Numerical Methods in Engineering\linebreak with python.} Cambrige University Press \textbf{37} (2010).
\item[2.] L.Héctor Juaréz V., \textit{Análisis Númerico} Universidad Autónoma Metropolitana \textbf{2008} (2010).
\item[3.] W. Kincaid, D. Cheney, \textit{Métodos Númericos y Computación,} sexta edición. Cengage Learning, 300-305.

\item[4.] Walter Mora F.,
\textit{Introducción a los Métodos Numéricos,} primera edición. Instituto Tecnológico de Costa Rica, 2013.

\item[5.] Dr. Herón Morales Marchena,
\textit{Interpolación, Diferenciación e Integración Numérica,} http://biblioteca.uns.edu.pe/saladocentes/archivoz/cu\\rzoz/cuaderno\_2.pdf.

\item[6.] Pontificia Universidad Católica del Perú, 2011
\textit{Química General,} http://corinto.pucp.edu.pe/quimicageneral\\/unidades-q2/unidad-2-cinetica-quimica.html.
%---------------------------------------------------------------------
%
\end{list}
\end{multicols}
\end{document}%\grid
